\documentclass[a4paper,10pt]{article}
\usepackage[utf8]{inputenc}

\usepackage{amsmath} 
% \usepackage[T1]{fontenc}
\usepackage{graphicx}
%\documentclass{article}
%\usepackage[showframe]{geometry}
\usepackage{layout}
\usepackage[margin=1.5in]{geometry}
%\usepackage{multirow}
\usepackage{hhline} %for double hlines
\usepackage{booktabs} %for all professional-looking table characterictics ;-)
\usepackage{float}
\restylefloat{table}
\floatstyle{plaintop}
\usepackage[tableposition=top]{caption}
\renewcommand{\arraystretch}{1.5}

\usepackage{subcaption}
\usepackage{listings}
\lstset{language=Python}
% Title Page
\title{\textbf{Python in the enterprise} \\ KNN-neighbour-based LHCb VELO sensor quality assesment}
\author{Dream team no. xxxx}

\begin{document}
\maketitle

\section{Introduction}
Feel free to store report ideas here...
\newline

===Milestone "0"===

Dziś spotkaliśmy się z drem.


1. Pierwszym krokiem będzie selekcja interesujących nas przykładowych danych. W tym celu wspólnie z Michałem przeglądniemy archiwum danych na serwerze w CERN-ie (zdaje się, że tylko my mamy tam konta...) i zdecydujemy co jest warte uwagi, (a następnie umieścimy te dane w miejscu dostępnym dla wszystkich.)

2. W kolejnym kroku należało będzie wyekstrahować z "suchych" danych odpowiednie parametry (jest to tzw. preprocessing). 
Chodzi o to, żeby każdy sensor / kanał odczytu miał konkretne parametry {x1, x2, ..., xn}, aby dało się zastosować machine learning (algortym KNN). Ponieważ raw-data są w plikach *.root, a ja już się bawiłem z tym środowiskiem, to zajmę się tym etapem. Wtedy uzyskane obrobione dane umieszczę w dostępnym dla wszystkich miejscu.

3. Algorytm KNN. Mając parametryzację {x_i} oraz informację, czy dany sensor jest "dobry", czy "zły" można wytrenować nasz algorytm i potem, jak Pan dr przygotuje specjalnie spreparowane "złe" próbki -- przetestujemy, czy nasze KNN działa.

Ponadto, okazało się, że zamiast typowej klasyfikacji binarnej (dobry vs zły) fajnie, jakby nasz algorytm zwracał prawdopodobieństwo, że sensor jest dobry (powiedzmy wartość zwracana Z jest z przedziału [0, 100], jeśli  Z >= 50 -> sensor dobry, gdy Z < 50 -> zły).
Tak więc, jest to regresja, nie klasyfikacja (albo, "klasyfikacja rozmyta", jak to dr określił).


Kroki 1-2 zajmą troszkę czasu. W międzyczasie warto, aby ktoś / kilka osób nabrało wprawy w stosowaniu KNN.
Osobiście sądzę, że warto inspirować się proponowaną książką oraz oprzeć projekt na scikit-learn, przy wykorzystaniu numpy.

Czy sądzicie, że już teraz warto przydzielić konkretne zadania, czy poczekać na pkty 1-2?
Oczywiście, jeśli kogoś interesuje wgląd w selekcje i preprocessing danych to dajcie znać.

Piszcie uwagi.

\end{document}          
